%\VignetteIndexEntry{segmentsSeq: Methylation locus identification}
%\VignettePackage{segmentSeq}

\documentclass[a4paper]{article}

%\usepackage{rotating}

\title{segmentSeq: methods for detecting methylation loci and differential methylation}
\author{Thomas J. Hardcastle}

\RequirePackage{/applications/R/R-3.2.4/library/BiocStyle/resources/tex/Bioconductor}

\AtBeginDocument{\bibliographystyle{/applications/R/R-3.2.4/library/BiocStyle/resources/tex/unsrturl}}

\usepackage{Sweave}
\begin{document}

\maketitle

\section{Introduction}

This vignette introduces analysis methods for data from high-throughput sequencing of bisulphite treated DNA to detect cytosine methylation. The \verb'segmentSeq' package was originally designed to detect siRNA loci \cite{Hardcastle:2011} and many of the methods developed for this can be used to detect loci of cytosine methylation from replicated (or unreplicated) sequencing data.

\section{Preparation}

Preparation of the segmentSeq package proceeds as in siRNA analysis. We begin by loading the \verb'segmentSeq' package.

\begin{Schunk}
\begin{Sinput}
>   library(segmentSeq)
\end{Sinput}
\end{Schunk}

Note that because the experiments that \verb'segmentSeq' is designed to analyse are usually massive, we should use (if possible) parallel processing as implemented by the \verb'parallel' package. If using this approach, we need to begin by define a \textsl{cluster}. The following command will use eight processors on a single machine; see the help page for 'makeCluster' for more information. If we don't want to parallelise, we can proceed anyway with a \verb'NULL' cluster. Results may be slightly different depending on whether or not a cluster is used owing to the non-deterministic elements of the method.

\begin{Schunk}
\begin{Sinput}
> if(require("parallel")) 
+ {
+     numCores <- min(8, detectCores())
+     cl <- makeCluster(numCores)
+ } else {
+     cl <- NULL
+ }
\end{Sinput}
\end{Schunk}

The \verb'segmentSeq' package is designed to read in output from the YAMA (Yet Another Methylome Aligner) program. This is a perl-based package using either bowtie or bowtie2 to align bisulphite treated reads (in an unbiased manner) to a reference and identify the number of times each cytosine is identified as methylated or unmethylated. Unlike most other aligners, YAMA does not require that reads that map to more than one location are discarded, instead it reports the number of alternate matches to the reference for each cytosine. This is then used by \verb'segmentSeq' to weight the observed number of methylated/un-methylated cytosines at a location.

\begin{Schunk}
\begin{Sinput}
> datadir <- system.file("extdata", package = "segmentSeq")
> files <- c("short_18B_C24_C24_trim.fastq_CG_methCalls",
+ "short_Sample_17A_trimmed.fastq_CG_methCalls",
+ "short_13_C24_col_trim.fastq_CG_methCalls",
+ "short_Sample_28_trimmed.fastq_CG_methCalls")
> mD <- readMeths(files = files, dir = datadir,
+ libnames = c("A1", "A2", "B1", "B2"), replicates = c("A","A","B","B"),
+ nonconversion = c(0.004777, 0.005903, 0.016514, 0.006134))
\end{Sinput}
\end{Schunk}

We can begin by plotting the distribution of methylation for these samples. The distribution can be plotted for each sample individually, or as an average across multiple samples. We can also subtract one distribution from another to visualise patterns of differential methylation on the genome.

\begin{Schunk}
\begin{Sinput}
> par(mfrow = c(2,1))
> dists <- plotMethDistribution(mD, main = "Distributions of methylation", chr = "Chr1")
> plotMethDistribution(mD, subtract = rowMeans(sapply(dists, function(x) x[,2])), main = "Differences between distributions", chr = "Chr1")
\end{Sinput}
\end{Schunk}

%\begin{sidewaysfigure}[!ht]
\begin{figure}[!ht]
\begin{center}
\includegraphics{methylationAnalysis-figMethDist}
\caption{Distributions of methylation on the genome (first two million bases of chromosome 1.}
\label{fig:Seg}
\end{center}
\end{figure}
%\end{sidewaysfigure}


Next, we process this \verb'alignmentData' object to produce a \verb'segData' object. This \verb'segData' object contains a set of potential segments on the genome defined by the start and end points of regions of overlapping alignments in the \verb'alignmentData' object. It then evaluates the number of tags that hit in each of these segments.

\begin{Schunk}
\begin{Sinput}
> sD <- processAD(mD, gap = 300, squeeze = 10, filterProp = 0.05, verbose = TRUE, strandSplit = TRUE, cl = cl)
\end{Sinput}
\end{Schunk}

We can now construct a segment map from these potential segments.

\subsection*{Segmentation by heuristic Bayesian methods}

A fast method of segmentation can be achieved by assuming a binomial distribution on the data with an uninformative beta prior, and identifying those potential segments which have a sufficiently large posterior likelihood that the proportion of methylation exceeds some critical value.

\begin{Schunk}
\begin{Sinput}
> hS <- heuristicSeg(sD = sD, aD = mD, prop = "auto", cl = cl, gap = 100, getLikes = FALSE)